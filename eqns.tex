\documentclass{article}

\usepackage{amsmath}

\usepackage{amsfonts}
\DeclareMathOperator\arctanh{arctanh}

\title{stuff that needs math formatting}
\author{Sam Brunacini}
\date{June 19, 2023}

\begin{document}

\maketitle

\section{number of layers needed}
\textit{Goal:} determine the number of layers $N$ needed to cover at least the proportion $k$ of the unit disk with a $(2,p,q)$ regular hyperbolic tiling.\\

$$f(\bold{x}):=\ln\left(\frac{1+||\bold{x}||}{1-||\bold{x}||}\right)$$
Equivalently, $f(\bold{x})=2\arctanh(||\bold{x}||)$.\\
If $\bold{x}$ is a number in the open unit disk in $\mathbb{C}$, $f(\bold{x})$ gives the hyperbolic distance of $\bold{x}$ to the origin in the Poincaré disk model. If $\bold{x}$ is a real number, $f(\bold{x})$ gives the hyperbolic distance to a point of Euclidean norm $\bold{x}$.

Suppose the tiling is generated from a rotationally symmetric regular hyperbolic polygon centered at the origin. Then the distance $d$ of each vertex of this polygon is uniquely determined by $p$ and $q$ due to congruence of all similar triangles in hyperbolic space. $N$ can be computed by taking the following ratio:
$$ \frac{f(k)}{f(d)} $$


\end{document}